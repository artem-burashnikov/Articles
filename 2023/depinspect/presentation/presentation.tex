\documentclass{beamer}
%Для защит онлайн лучше использовать разрешение 16x9
% \documentclass[aspectratio=169]{beamer}

% !TeX spellcheck = ru_RU
% !TEX root = artem-burashnikov_parallel_bfs_experiments.tex
% Опциональные добавления используемых пакетов. Вполне может быть, что они вам не понадобятся, но в шаблоне приведены примеры их использования.
\usepackage{tikz} % Мощный пакет для создание рисунков, однако может очень сильно замедлять компиляцию
\usetikzlibrary{decorations.pathreplacing,calc,shapes,positioning,tikzmark}

% Библиотека для TikZ, которая генерирует отдельные файлы для каждого рисунка
% Позволяет ускорить компиляцию, однако имеет свои ограничения
% Например, ломает пример выделения кода в листинге из шаблона
% \usetikzlibrary{external}
% \tikzexternalize[prefix=figures/]

\newcounter{tmkcount}

\tikzset{
  use tikzmark/.style={
    remember picture,
    overlay,
    execute at end picture={
      \stepcounter{tmkcount}
    },
  },
  tikzmark suffix={-\thetmkcount}
}

\usepackage{booktabs} % Пакет для верстки "более книжных" таблиц, вполне годится для оформления результатов
% В шаблоне есть команда \multirowcell, которой нужен этот пакет.
\usepackage{multirow}
\usepackage{siunitx} % для таблиц с единицами измерений
\usepackage{algorithm}
\usepackage{amsmath}
\usepackage{url}

\newcommand{\cd}[1]{\texttt{#1}}
\newcommand{\inbr}[1]{\left<#1\right>}

% Для названий стоит использовать \textsc{}
\newcommand{\OCaml}{\textsc{OCaml}}
\newcommand{\miniKanren}{\textsc{miniKanren}}
\newcommand{\BibTeX}{\textsc{BibTeX}}
\newcommand{\vsharp}{\textsc{V$\sharp$}}
\newcommand{\fsharp}{\textsc{F$\sharp$}}
\newcommand{\csharp}{\textsc{C$\sharp$}}
\newcommand{\GitHub}{\textsc{GitHub}}
\newcommand{\SMT}{\textsc{SMT}}

\newcolumntype{L}[1]{>{\raggedright\let\newline\\\arraybackslash\hspace{0pt}}m{#1}}
%\newcolumntype{C}[1]{>{\centering\let\newline\\\arraybackslash\hspace{0pt}}m{#1}}
\newcolumntype{R}[1]{>{\raggedleft\let\newline\\\arraybackslash\hspace{0pt}}m{#1}}

%  Команды и пакеты, не используемые в шаблоне, которые тем не менее могут быть полезными.

% \newcolumntype{Y}{>{\centering\arraybackslash}X}

% \usepackage{mathrsfs}

%\lstdefinelanguage{ocaml}{
%    keywords={@type, function, fun, let, in, match, with, when, class, type,
%    nonrec, object, method, of, rec, repeat, until, while, not, do, done, as, val, inherit, and,
%    new, module, sig, deriving, datatype, struct, if, then, else, open, private, virtual, include, success, failure,
%    lazy, assert, true, false, end},
%    sensitive=true,
%    commentstyle=\small\itshape\ttfamily,
%    keywordstyle=\ttfamily\bfseries, %\underbar,
%    identifierstyle=\ttfamily,
%    basewidth={0.5em,0.5em},
%    columns=fixed,
%    fontadjust=true,
%    literate={->}{{$\to$}}3 {===}{{$\equiv$}}1 {=/=}{{$\not\equiv$}}1 
%    {|>}{{$\triangleright$}}3 {\\/}{{$\vee$}}2 
%    {/\\}{{$\wedge$}}2 {>=}{{$\ge$}}1 {<=}{{$\le$}} 1,
%    morecomment=[s]{(*}{*)}
%}

\usepackage{upquote}

\usepackage{color}
\definecolor{bluekeywords}{rgb}{0.13,0.13,1}
\definecolor{greencomments}{rgb}{0,0.5,0}
\definecolor{redstrings}{rgb}{0.9,0,0}

\lstdefinelanguage{pseudoFSharp}%
    {morekeywords={def, new, match, with, recursive, open, module, namespace, type, of, member, % 
    and, for, while, true, false, in, do, begin, end, fun, function, return, yield, try, %
    mutable, if, then, else, cloud, async, static, use, abstract, interface, inherit, finally },
    otherkeywords={ let!, return!, do!, yield!, use!, var, from, select, where, call, by },
    keywordstyle=\color{redstrings},
    sensitive=true,
    tabsize=2,
    comment=[l][\color{greencomments}]{//}
}

\definecolor{codegreen}{rgb}{0,0.6,0}
\definecolor{codegray}{rgb}{0.5,0.5,0.5}
\definecolor{codepurple}{rgb}{0.58,0,0.82}
\definecolor{backcolour}{rgb}{1.0,1.0,1.0}

\lstdefinestyle{codelistingstyle}{
    backgroundcolor=\color{backcolour},   
    commentstyle=\color{codegreen},
    keywordstyle=\color{magenta},
    numberstyle=\small\color{codegray},
    stringstyle=\color{codepurple},
    basicstyle=\ttfamily\small,
    breakatwhitespace=false,         
    breaklines=true,                 
    captionpos=b,                    
    keepspaces=true,                 
    numbers=left,                    
    numbersep=5pt,                  
    showspaces=false,                
    showstringspaces=false,
    showtabs=false,                  
    tabsize=2,
    commentstyle=\color{codegreen}
}

\let\sq=\textquotesingle


% То, что в квадратных скобках, отображается внизу по центру каждого слайда.
\title[Метаданные пакетов в Linux]{Проектирование и создание системы для анализа зависимостей пакетов в дистрибутивах GNU/Linux}

% То, что в квадратных скобках, отображается в левом нижнем углу.
\institute[СПбГУ]{}

% То, что в квадратных скобках, отображается в левом нижнем углу.
\author[Бурашников Артем]{Бурашников Артем Максимович, группа 22.Б07-мм}

\begin{document}
{
\setbeamertemplate{footline}{}
% Лого университета или организации, отображается в шапке титульного листа
\begin{frame}
	\includegraphics[width=1.4cm]{pictures/SPbGU_Logo.png}
	\vspace{-35pt}
	\hspace{-10pt}
	\begin{center}
		\begin{tabular}{c}
			\scriptsize{Санкт-Петербургский государственный университет} \\
			\scriptsize{Кафедра системного программирования}
		\end{tabular}
		\titlepage
	\end{center}

	\btVFill

	{\scriptsize
		% У научного руководителя должна быть указана научная степень
		\textbf{Научный руководитель:} К.К. Смирнов, старший преподаватель кафедры ИАС \\
		% Консультанта может и не быть. Должна быть указана должность или ученая степень
		% \textbf{Консультант:}  П.П. Петров, программист ЗАО \enquote{Компания с ну очень-очень-очень длинным названием}\\
		% Для курсовой не обязателен. Должна быть указана должность или ученая степень
		% \textbf{Рецензент:} д.т.н., проф. И.И. Иванов, исполнительный директор ООО \enquote{Рога и копыта}
	}
	\begin{center}
		\vspace{5pt}
		\scriptsize{Санкт-Петербург\\
			2023}
	\end{center}

\end{frame}
}

\begin{frame}[fragile]
	\frametitle{Введение}
	\begin{itemize}
		\item Вокруг каждого семейства дистрибутивов GNU/Linux своя экосистема
		\item Для установки программ используются пакеты, чьи форматы не унифицированы
		\item Хотим анализировать метаданные пакетов в контексте архитектуры \texttt{RISC-V} --- может быть полезно как для бизнеса, так и для разработчиков
		\item Междистрибутивный и межархитектурный анализ --- проблема
	\end{itemize}
\end{frame}

% Обязательный слайд: четкая формулировка цели данной работы и постановка задачи
% Описание выносимых на защиту результатов, процесса или особенностей их достижения и т.д.
\begin{frame}
	\frametitle{Постановка задачи}
	\textbf{Цель}: создание системы, которая помогает проводить междистрибутивный и межархитектурный анализ пакетов и их зависимостей в ОС \texttt{Linux} (приоритетная архитектура --- \texttt{RISC-V}).
	\newline
	\newline
	\textbf{Задачи}:
	\begin{itemize}
		\item Провести обзор инструментов, позволяющих анализировать метаданные пакетов, с целью выбора функциональности создаваемого инструмента
		\item Проведя анализ форматов метаданных, выбрать два дистрибутива, для которых обозначенная функциональность будет реализована
		\item Реализовать приложение на языке \texttt{Python}, учитывая сформированные требования
	\end{itemize}
\end{frame}

\begin{frame}
	\frametitle{Обзор существующих инструментов}
	\textbf{Цель обзора} --- выбрать требуемую для реализации функциональность
	\newline
	\newline
	\textbf{Критерии отбора инструментов:}
	\begin{itemize}
		\item Поддерживается разработчиками
		\item Родной для дистрибутива или кросс-платформенный
		\item Может выводить информацию о метаданных для нескольких архитектур и/или дистрибутивов
	\end{itemize}

\end{frame}

\begin{frame}
	\frametitle{Обзор существующих инструментов --- cравнение}
	\begin{itemize}
		\item \textbf{Apt} --- пакетный менеджер для Debian и производных систем
		\item \textbf{Pactree} --- пакетный менеджер для Arch Linux
		\item \textbf{Debtree} --- сторонняя утилита, распространяемая через apt
		\item \textbf{Repology} --- веб-приложение агрегатор
	\end{itemize}

	\begin{table}[ht]
		\centering
		\begin{tabular}{|c|c|c|c|c|}
			\hline
			& Apt & Pactree & Debtree & Repology \\
			\hline
			Прямые зависимости & \checkmark & \checkmark & \checkmark & \checkmark \\
			\hline
			Обратные зависимости & \checkmark & \checkmark & \checkmark & \checkmark \\
			\hline
			Сравнение метаданных & \ding{55} & \ding{55} & \ding{55} & \ding{55} \\
			\hline
			Построение графа & \ding{55} & \checkmark & \checkmark & \ding{55} \\
			\hline
			Кросс-платформенность & \ding{55} & \ding{55} & \ding{55} & \checkmark \\
			\hline
			Интерфейс & CLI & CLI & CLI & REST API \\
			\hline
		\end{tabular}
	\end{table}
\end{frame}

% \begin{frame}
% 	\frametitle{Обзор существующих инструментов --- вывод}
% 	\begin{itemize}
% 		\item Сравнение по прямым зависимостям пакетов между различными архитектурами
% 		\item Вывод имен пакетов и архитектур, для которых можно произвести сравнение
% 		\item Агрегация пакетов из различных дистрибутивов
% 		\item Пользовательский интерфейс
% 		\item Кросс-платформенность
% 	\end{itemize}
% \end{frame}

\begin{frame}
	\frametitle{Формат пакетов}
	Выделены две основные категории пакетов:
	\begin{itemize}
		\item \textbf{rpm}: кодируют спецификации в бинарной форме
		\item \textbf{deb}: текстовое представление
	\end{itemize}

	Несмотря на различия форматов архивов, спецификации зависимостей в метаданных практически идентичны
\end{frame}

\begin{frame}
	\frametitle{Спецификация зависимостей}
	Отношения, используемые между пакетами, на примере спецификации \texttt{Debian}:
	\begin{enumerate}
		\item Depends
		\item Recommends
		\item Suggests
		\item Enhances
		\item Pre-Depends
		\item Conflicts
		\item Provides
		\item Replaces
	\end{enumerate}
\end{frame}

\begin{frame}
	\frametitle{Выбор первичных дистрибутивов}
	\begin{itemize}
		\item \textbf{Ubuntu}: метаданные в текстовом виде, простой синтаксис --- позволило создать прототип и разработать интерфейс для пользователя
		\item \textbf{Fedora}: метаданные в репозитории представлены в файле \textbf{.sqlite}, использует пакеты rpm --- созданное приложение работает с обоими форматами
		      % \item Обе ОС поддерживают \texttt{RISC-V} и входят в десятку самых популярных, согласно данным портала \texttt{DistroWatch}
		      % \item Помимо \texttt{RISC-V}, релизы и upstream репозитории выбраны произвольно
		      % \item Учтена возможность расширить приложение добавлением других релизов, репозиториев и дистрибутивов.
	\end{itemize}
\end{frame}

% \begin{frame}[fragile]
% 	\frametitle{Использованные технологии}
% 	Реализовано на \texttt{Python} с использованием следующих библиотек и приложений:
% 	\begin{itemize}
% 		\item \textbf{Poetry}: менеджер зависимостей и система сборки
% 		\item \textbf{black}: форматтер для поддержания единого стиля кода
% 		\item \textbf{ruff}: линтер, использующий те же правила, что и \texttt{flake8}
% 		\item \textbf{mypy}: статическая проверка типов, для всех функций написаны аннотации типов
% 	\end{itemize}

% 	Для этих библиотек настроен \texttt{CI}
% \end{frame}

%% Совместить два слайду внизу в один!
%
% \begin{frame}[fragile]
% 	\frametitle{Реализация --- базы данных}
% 	Приложение хранит метаданные в базах данных \texttt{SQLite}
% 	\begin{itemize}
% 		\item Упростило процесс внедерения метаданных \texttt{Fedora}
% 		\item Позволило переиспользовать метаданные, храня их на диске
% 		\item Позволило реализовать создание нетривиальных запросов с помощью \texttt{SQL}
% 	\end{itemize}
% \end{frame}

\begin{frame}[fragile]
	\frametitle{Реализация --- базы данных}
	\begin{figure}
		\centering
		\includesvg[width=\textwidth]{pictures/bd-schema}
	\end{figure}

	\begin{itemize}
		% \item Упростило процесс внедерения метаданных \texttt{Fedora}
		\item Позволило переиспользовать метаданные, храня их на диске
		\item Позволило реализовать создание нетривиальных запросов с помощью \texttt{SQL}
	\end{itemize}
\end{frame}
%

\begin{frame}[fragile]
	\frametitle{Реализация --- CLI}
	Для конечного пользователя с помощью библиотеки \texttt{Click} создано консольное приложение
	\begin{itemize}
		\item Реализация \texttt{CLI} требует значительно меньше времени, чем написание графического интерфейса
		\item Простота добавления новых команд
		\item Удовлетворяет требованиям
	\end{itemize}

	\begin{figure}[ht]
		\centering
		\includesvg[width=0.8\textwidth]{pictures/cli}
		\label{cli}
	\end{figure}
\end{frame}

\begin{frame}
	\frametitle{Пример использования}
	\begin{figure}[ht]
		\raggedright
		\includesvg[width=0.55\textwidth]{pictures/result}
	\end{figure}
\end{frame}

\begin{frame}[fragile]
	\frametitle{Реализация --- архитектура}
	\begin{figure}[ht]
		\centering
		\includesvg[width=0.8\textwidth]{pictures/packages-uml}
	\end{figure}
\end{frame}

% \begin{frame}[fragile]
% 	\frametitle{Реализация --- архитектура}
% 	Детализация модуля \texttt{distributions}
% 	\begin{figure}[ht]
% 		\centering
% 		\includesvg[width=0.7\textwidth]{pictures/distributions-detailed}
% 	\end{figure}
% \end{frame}

% \begin{frame}[t]
% 	\frametitle{Реализация --- архитектура}
% 	Независимые компоненты:
% 	\begin{itemize}
% 		\item \textbf{CLI} --- интерфейс для пользователя
% 		\item \textbf{Fetcher} --- загрузка метаданных, указанных в конфигурационном файле проекта \texttt{pyproject.toml}
% 		\item \textbf{Extractor} --- обработка загруженных разного рода архивов, содержащих метаданные
% 		\item \textbf{Database} -- абстракция над запросами в базу данных с помощью языка запросов \texttt{SQL}
% 		\item \textbf{Printer} - вывод результата в консоль
% 	\end{itemize}

% 	Близко к архитектурному стилю \enquote{Каналы и Фильтры}
% \end{frame}

% \begin{frame}[t]
% 	\frametitle{Реализация --- ETL}
% 	Во время поступления команды от пользователя заранее неизвестно, какой порядок действий нужно совершить:
% 	\begin{itemize}
% 		\item Для инициализации метаданных \texttt{Ubuntu} нужно выполнить целую цепочку (загрузить архивы, произвести парсинг текста, создать и наполнить базу данных)
% 		\item Для \texttt{Fedora} --- загрузить и распаковать файлы \texttt{.sqlite}.
% 	\end{itemize}
% \end{frame}

\begin{frame}[t]
	\frametitle{Реализация --- ETL}
	% Во время поступления команды от пользователя заранее неизвестно, какой порядок действий нужно совершить:
	\begin{itemize}
		\item Для инициализации метаданных \texttt{Ubuntu} нужно выполнить целую цепочку (загрузить архивы, произвести парсинг текста, создать и наполнить базу данных)
		\item Для \texttt{Fedora} --- загрузить и распаковать файлы \texttt{.sqlite}.
	\end{itemize}

	% Таким образом:
	\begin{itemize}
		\item Создан метакласс \textbf{Package}, имеющий набор полей, соответствующих именам полей метаданных, и декларирующий абстрактные статические методы, сопоставляемые консольным командам
		\item Дочерние классы \texttt{Ubuntu} и \texttt{Fedora} реализуют эти статические методы по-своему
	\end{itemize}
\end{frame}

\begin{frame}
	\frametitle{Результаты}
	\begin{itemize}
		\item Проведён обзор четырех инструментов, способных работать с метаданными пакетов, на основе анализа функциональности которых созданы четыре доступные для конечного пользователя команды приложения
		\item Произведен анализ форматов метаданных и внедрены метаданные пакетов \texttt{Ubuntu} (6130 пакетов) и \texttt{Fedora} (65290 пакетов)
		\item На языке \texttt{Python} реализовано приложение\footnote{Репозиторий приложения: \href{https://github.com/artem-burashnikov/depinspect}{https://github.com/artem-burashnikov/depinspect}} с модульной архитекторой и консольным интерфейсом, включающее два дистрибутива и позволяющее производить межархитектурный анализ зависимостей пакетов
	\end{itemize}
\end{frame}

%\addtocounter{framenumber}{1}
% \appendix

% \begin{frame}
% 	\frametitle{Пример использования}
% 	\begin{figure}[ht]
% 		\raggedright
% 		\includesvg[width=0.55\textwidth]{pictures/result}
% 	\end{figure}
% \end{frame}

\end{document}
