% !TeX spellcheck = ru_RU
% !TEX root = vkr.tex

\section{Обзор}
\label{sec:relatedworks}

В данном разделе нужно описать всё, что необходимо для понимания Вашей работы и что придумали не Вы.
В дальнейших разделах нельзя прерывать повествование, например, для рассказа о деталях используемой технологии или архитектуре старой системы, потому что читателю будет трудно отличить Ваш вклад от не Вашего.

Любой обзор пишется с какой-то целью (обосновать актуальность, найти и описать интересные решения, сравнить и выбрать технологии) и по какой-то методике поиска материала (например, поиск N релевантных статей на таких-то сервисах).
Не будет лишним это всё явно описать.

\subsection{Обзор существующих решений}

\emph{Обзор существующих решений должен быть.}
Здесь нужно писать, что индустрия и наука уже сделали по вашей теме.
Он нужен, чтобы Вы случайно не изобрели какой-нибудь велосипед.

По-английски называется related works или previous works.

Если Ваша работа является развитием предыдущей и плохо понима\-ема без неё, то обзор должен идти в начале.
Если же Вы решаете некоторую задачу новым интересным способом, то если поставить обзор в начале, то читатель может устать, пока доберется до вашего решения.
В этом случае уместней поставить обзор после описания Вашего подхода к проблеме%
\footnote{Такой подход рекомендуется в работе~\cite{SPJGreatPaper}.
    Вполне возможно, что Ваш реальный научный руководитель будет не согласен, и потребует, чтобы обзор был в начале.}.

В обзоре вам нужно рассказать про \emph{преимущества и недостатки} того, что было сделано до Вас.
Неправильным будет перечислять только недостатки, так как если Ваша работа хоть где-то хуже предыдущей, то рецензент будет радостно потирать руки и заваливать Вашу работу.
Гораздо лучше, если Вы честно признаетесь в этом сами.

\subsection{Обзор используемых технологий}

Для технических работ обзор может обозревать продукт, в рамках которого Вы выполняете задачу, другие продукты, где решалась схожая задача, а также используемые технологии с обоснованием выбора тех, которые Вы дальше используете.
Это всё, скорее всего, будет отдельными подразделами обзора.
\enquote{Выбор} подразумевает наличие вариантов, поэтому опишите, из чего выбирали и почему выбрали то, что выбрали.
Очень желательны чёткие критерии сравнения и сводная таблица в конце, где стоят плюсы и минусы рядом с каждым рассматриваемым вариантом.

В обзоре необходимо ссылаться на работы других людей.
В данном шаблоне задумано, что литература будет указываться в файле \verb=vkr.bib=.
В нём указываются пункты литературы в формате \BibTeX{}, а затем на них можно ссылаться с помощью \verb=\cite{...}=.
Та литература, на которую Вы сошлетесь, попадет в список литературы в конце документа.
Если не сошлетесь~---  не попадёт.
Спецификацию в формате \BibTeX{} почти никогда (для второго курса~--- никогда), не нужно придумывать руками.
Правильно: находить в интернете описание цитируемой статьи%
\footnote{Например, \url{https://dl.acm.org/doi/10.1145/3408995} (дата обращения: \DTMdate{2022-12-17}).},
копировать цитату с помощью кнопки \foreignquote{english}{Export Citation} и вставлять в \BibTeX{} файл.
Так же умеет генерировать \BibTeX{}-описания и Google Scholar%
\footnote{Поисковая система для научных текстов Google Scholar, \url{https://scholar.google.com} (дата обращения: \DTMdate{2024-01-13})}.
Если так не делать, то оформление литературы будет обрастать ошибками.
Например, \BibTeX{} по особенному обрабатывает точ\-ки, запятые и \verb=and= в списке авторов, что позволяет ему самому понимать, сколько авторов у статьи, и что там фамилия, что~--- имя, а что~--- отчество.
Google Scholar пытается генерировать описания автоматически, так что, возможно, потребуется ручная правка~--- обязательно проверьте свой список литературы.

В обзоре и в остальном тексте вы наверняка будете использовать названия продуктов или языков программирования (например, \csharp{}).
Для них рекоменду\-ется (в файле \verb=preamble2.tex=) за\-дать специальные команды, чтобы писать сложные названия правильно и одинаково по всему доку\-менту.
Написать с ошибкой  название любимого языка программирова\-ния науч\-ного руко\-водителя~--- идеальный вариант его разозлить.

\subsection{Выводы}

Опишите явно, что читатель должен был вынести из обзора в отдельном подразделе.
