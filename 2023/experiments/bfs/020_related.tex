% !TeX spellcheck = ru_RU
% !TEX root = vkr.tex

\section{Сопутствующие исследования}
\label{sec:relatedworks}
Алгоритм обхода графа в ширину ввиду своей прикладной значимости был проанализирован в различном контексте в ряде исследовательских работ. 

В \cite{adaptiveBFS} отмечается, что производительность BFS в значительной мере зависит от топологии подаваемого на вход графа. Авторы показали, что в случае большого количества итераций алгоритма при малых количествах вершин/ребер между итерациями (то есть малом количестве вершин во фронте на каждой итерации) параллельная версия терпит существенное снижение производительности из-за накладных расходов.

Кроме того, в \cite{scalableBFS} продемонстрировано, что последовательная версия алгоритма во многих ситуациях оказывается предпочтительнее не оптимизированной параллельной ввиду большой задержки работы с памятью и высокой вычислительной стоимостью её синхронизации. 

Многие авторы находят решение упомянутых проблем в тонкой настройке взаимодействия с общей памятью в используемой архитектуре или применении адаптивных алгоритмов, способных динамически контролировать количество используемых потоков во время исполнения.

Обширный список проведенных исследований позволяет с высокой точностью прогнозировать зависимости между параметрами входных данных, выбранной архитектурой и ожидаемой производительностью используемой реализации обхода в ширину. Данная работа будет посвящена выявлению таких зависимостей для алгоритма BFS, реализованного с применением методов линейной алгебры, что существенным образом влияет не только на сам алгоритм, но и на внутреннее представление графа в памяти компьютера, в связи с чем полученные результаты могут представлять особый интерес.
