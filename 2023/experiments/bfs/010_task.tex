% !TeX spellcheck = ru_RU
% !TEX root = vkr.tex

\section{Постановка задачи}
\label{sec:task}
 Целью работы является провести экспериментальное исследование производительности обхода в ширину и ответить на следующие вопросы: 
\begin{enumerate}
    \item При каких параметрах графа выгоднее использовать параллельную версию алгоритма, а при каких последовательную?
    \item Использование какого количества потоков даёт наибольший выигрыш в производительности и почему?
\end{enumerate}

 Для её выполнения были постав\-лены следующие задачи:
 \begin{itemize}
 \item  Реализовать серию измерений производительности последовательной и параллельной версий обхода в ширину с контролируемым набором параметров графа, таких как количество вершин, степень вершин и общая плотность структуры.
 \item  Оценить влияние конкретных характеристик графа на итоговую производительность BFS.
 \item  Провести эксперименты с разным количеством потоков и измерить производительность алгоритма обхода в ширину при каждом сценарии, обращая внимание на накладные расходы при синхронизации и доступе к общей памяти.
 \end{itemize}
