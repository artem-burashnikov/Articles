% !TeX spellcheck = ru_RU
% !TEX root = artem-burashnikov_parallel_bfs_experiments.tex

\section{Постановка задачи}
\label{sec:task}
 Целью работы является провести экспериментальное исследование производительности обхода в ширину и ответить на следующие вопросы. 
\begin{enumerate}
    \item При каких параметрах графа выгоднее использовать параллельную версию алгоритма, а при каких последовательную?
    \item Использование какого количества потоков даёт наибольший выигрыш в производительности и почему?
\end{enumerate}

 Для её выполнения были постав\-лены перечисленные ниже задачи.
 \begin{itemize}
 	\item Реализовать параллельную и последовательную версии алгоритма обхода в ширину с использованием структур, подходящих для хранения в памяти компьютера разреженных матриц и векторов.
 	\item  Оценить влияние конкретных характеристик графа и количества используемых потоков на итоговую производительность BFS и найти критические величины этих параметров.
 \end{itemize}

Выполнение поставленных задач позволит определить какая версия алгоритма (последовательная или параллельная) предпочтительнее к использованию при том или ином сценарии.
