% !TeX spellcheck = ru_RU
% !TEX root = burashnikov_depinspect.tex

\section{Обзор предметной области}
\label{sec:relatedworks}
Для проектирования системы с целью выявления желаемой функциональности необходимо ознакомиться с инструментами, предоставляющими анализ пакетов.
Кроме того, нужно выбрать первоначальные дистрибутивы и проанализировать форматы, в которых представлены метаданные. В конце раздела указаны использованые сторонние библиотеки.

\subsection{Существующие аналоги}
В конкретном дистрибутиве используется своё программное обеспечение, позволяющее провести требуемый анализ.
Далее представлены некоторые из них.

\subsubsection{\texttt{Apt}}
\texttt{Apt}\footnote{\href{https://manpages.debian.org/stretch/apt/apt.8.en.html\#SCRIPT_USAGE_AND_DIFFERENCES_FROM_OTHER_APT_TOOLS/}{Документация apt. Дата обращения: \DTMdate{2023-12-13}}} --- это пакетный менеджер, по умолчанию поставляемый вместе с дистрибутивом {\debian} и производными системами.
Согласно документации, \texttt{apt} является высокоуровневой оболочкой для других инструментов, поэтому для просмотра предоставляемой функциональности нужно обратиться к \texttt{apt-cache}\footnote{\href{https://manpages.debian.org/stretch/apt/apt-cache.8.en.html}{Документация \texttt{apt-cache}. Дата обращения: \DTMdate{2023-12-13}}}.
Следующие возможности утилиты согласуются с целью работы:
\begin{enumerate}
	\item \texttt{depends} отображает список каждой зависимости, которую имеет пакет, а также все возможные другие пакеты, которые могут удовлетворить эту зависимость;
	\item \texttt{rdepends} отображает список пакетов, для которых данный является зависимостью;
	\item \texttt{pkgnames} выводит имена пакетов, о которых знает \texttt{apt-cache}.
\end{enumerate}

Одним из существенных минусов \texttt{apt} является то, что программу не имеет смысла использовать на других дистрибутивах, потому что менеджер плотно интегрирован с системами, под которые создан.

\subsubsection{\texttt{Pactree}}
\texttt{Pactree}\footnote{\href{https://man.archlinux.org/man/extra/pacman-contrib/pactree.8.en}{Документация pactree. Дата обращения: \DTMdate{2023-12-13}}} --- утилита для дистрибутива \textsc{Arch Linux}, позволяющая выводить на экран деревовидную структуру зависимостей для пакетов.
Из функциональности отмечаются те же пункты, что и для \texttt{apt}. Дополнительно имеется возможность вывода результата запроса в формате \texttt{Graphviz}\footnote{\href{https://graphviz.org/}{Библиотека Graphviz для визуализации графов}} для последующей визуализации полученного графа.

Минус \texttt{pactree} заключается в том, что на момент написания работы \textsc{Arch~Linux} официально не поддерживает набирающую популярность и активно развивающуюся архитектуру \textsc{RISC-V}~\cite{RISCVSurvey}.

\subsubsection{\texttt{Debtree}}
\texttt{Debtree}\footnote{\href{https://manpages.ubuntu.com/manpages/xenial/man1/debtree.1.html}{Документация debtree на ресурсе {\ubuntu}. Дата обращения: \DTMdate{2023-12-13}}} --- инструмент для {\debian} и производных систем, позиционирующий себя как \enquote{граф зависимостей на стероидах}. Функциональность сосредоточена на построении графа.
Выводит результет на языке \texttt{dot}, читаемом утилитой \texttt{Graphviz}. Возможности аналогичны уже обозначенным приложениям, однако обладает дополнительными опциями для построения графов (например, цвет, глубина и т.д.).
Отдельно стоит отметить возможность при вызове команды добавить опциональный аргумент \textbf{arch}, указав архитектуру, зависимость пакета на которой требуется показать.

К сожалению, \texttt{Debtree} можно использовать только на производных {\debian}.
Кроме того, несмотря на то, что можно указать архитектуру, без дополнительных системных настроек будет использована архитектура системы, на которой запущено приложение, вне зависимости от того, какая архитектура указана аргументом.

\subsubsection{\texttt{Repology}}
\texttt{Repology}\footnote{\href{https://repology.org/docs/about}{Информация о ресурсе Repology. Дата обращения: \DTMdate{2023-12-13}}} в отличие от уже рассмотренных приложений напрямую не позиционируется как система анализа пакетов.
Проект является агрегатором репозиториев и смежных ресурсов, предоставляющих установочные архивы. \texttt{Repology} имеет обширный список возможностей для анализа версий библиотек, наличия или отсутствия их в репозиториях.
Согласно описанию, сервис отслеживает и позволяет сравнивать версии пакетов из более чем 120 репозиториев.
Хотя \texttt{Repology} и не предоставляет никакой логики для сравннения зависимостей, особое внимание уделено некоторым аспектам технической реализации проекта в целом, потому что они интересны для проектируемой в рамках семестровой работы системы.

Во-первых, сервис представляет собой веб-приложение, разделенное на несколько независимых компонент: \textbf{webapp}\footnote{\href{https://github.com/repology/repology-webapp}{Репозиторий компоненты webapp сервиса Repology. Дата обращения: \DTMdate{2023-12-13}}}, \textbf{updater}\footnote{\href{https://github.com/repology/repology-webapp}{Репозиторий компоненты updater сервиса Repology. Дата обращения: \DTMdate{2023-12-13}}} и \textbf{ruleset}\footnote{\href{https://github.com/repology/repology-webapp}{Репозиторий компоненты ruleset сервиса Repology. Дата обращения: \DTMdate{2023-12-13}}}.
Во-вторых, сами метаданные скачиваются из архивов соответствующих дистрибутивам репозиториев (те же репозитории использует, например \texttt{apt}), затем сервис производит парсинг загруженных метаданных и сохранение их в базах данных \textsc{SQLite}.
Указанные особенности реализации оказались интересны для проектируемой системы, так как позволили внедрить модульный дизайн и обеспечить переиспользование полученной из репозиториев информации об архивах.

\subsection{Результаты анализа аналогов}
В таблице~\ref{tbl:tool-comparison} представлены рассмотренные инструменты и основная функциональность каждого из них.

\begin{table}[ht]
	\centering
	\begin{tabular}{|c|c|c|c|c|}
		\hline
		& Apt & Pactree & Debtree & Repology \\
		\hline
		Прямые зависимости & \checkmark & \checkmark & \checkmark & \checkmark \\
		\hline
		Обратные зависимости & \checkmark & \checkmark & \checkmark & \checkmark \\
		\hline
		Сравнение метаданных & \ding{55} & \ding{55} & \ding{55} & \ding{55} \\
		\hline
		Построение графа & \ding{55} & \checkmark & \checkmark & \ding{55} \\
		\hline
		Кросс-платформенность & \ding{55} & \ding{55} & \ding{55} & \checkmark \\
		\hline
		API & CLI & CLI & CLI & REST API \\
		\hline
	\end{tabular}
	\caption{Сравнение инструментов.}
	\label{tbl:tool-comparison}
\end{table}

Таким образом, вывод для пакета его прямых и обратных метаданных --- основная функция, присущая всем утилитам такого рода.
Далее, не все рассмотренные приложения могут рисовать полный граф зависимостей и никакая из утилит не может напрямую проводить сравнение метаданных пакетов между различными архитектурами ни в рамках одного дистрибутива, ни между пакетами в рамках различных дистрибутивов.
Так, в основу созданного инструмента заложены следующие особенности:
\begin{enumerate}
	\item сравнение по прямым зависимостям пакетов между различными архитектурами в рамках одного дистрибутива;
	\item вывод имен пакетов и архитектур, для которых можно произвести сравнение;
	\item архитектура приложения должна быть гибкой и позволять расширять фукциональность, используя не только зависимости, но и другие потенциально значимые метаданные;
	\item агрегация пакетов из различных дистрибутивов;
	\item пользовательский интерфейс;
	\item кросс-платформенность инструмента.
\end{enumerate}

\subsection{Формат пакетов в {\linux}}
\label{package-formats}
Для дистрибутивов {\linux} существуют две основные категории пакетов: \texttt{rpm} и \texttt{deb}~\cite{LinuxDependencyFormat}.
Пакеты формата \texttt{rpm} в основном используются с \textsc{Red Hat}, {\fedora}, \textsc{Suse} и производными дистрибутивами в то время как пакеты \texttt{deb} применяются для {\debian} и его производных.
И \texttt{rpm}, и \texttt{deb} представляют собой сжатые архивы, содержащие все необходимые данные для корретного использования системой. Кроме того, они включают метаданные, описывающие атрибуты и требования, касающиеся окружения, в котором установливается или настраивется пакет.
Пакеты \texttt{rpm} кодируют спецификации в бинарной форме и включают их в состав своих архивных файлов, в то время как для \texttt{deb} используется текстовое представление, что делает их более удобными для обработки.
Несмотря на указанные различия, спецификации зависимостей в метаданных для обоих типов пакетов практически идентичны.

\subsection{Спецификация зависимостей}
Прежде всего необходимо понимать отношения, используемые между пакетами\footnote{\href{https://www.debian.org/doc/debian-policy/ch-relationships.html}{Руководство Debian. Дата посещения \DTMdate{2023-12-14}}}.
Ниже на примере спецификации формата метаданных в дистрибутиве {\debian} приведен список таких отношений с пояснениями.
\begin{enumerate}
	\item \textit{Depends}.\newline
	      Пакет \textbf{A} не будет настроен, пока все пакеты, перечисленные в этом поле не будут корректно настроены. Пакет может находиться в системе в ``не настроенном'' состоянии.
	\item \textit{Recommends}.\newline
	      Для пакета \textbf{A} в этом поле перечислены пакеты, которые обычно устанавливаются вместе c \textbf{A}.
	\item \textit{Suggests}.\newline
	      Перечисленные в этом поле пакеты связаны с пакетом \textbf{A} и, возможно, могут улучшить его полезность, но установка только \textbf{A} без них разумна и ничем не ограничена.
	\item \textit{Enhances}.\newline
	      Работает как \textit{Suggests}, но в обратную сторону: в этом поле указаны пакеты, для которых пакет \textbf{A} попадает в \textit{Suggests}, потенциально улучшая их полезные свойства.
	\item \textit{Pre-Depends}.\newline
	      Означает почти то же, что и \textit{Depends}, но дополнительно требует завершения установки указанных в поле пакетов до установки пакета \textbf{A}.
	\item \textit{Conflicts}.\newline
	      Пакеты, указанные в этом поле, не могут существовать в системе одновременно с пакетом \textbf{A}.
	\item \textit{Provides}.\newline
	      В этом поле указаны виртуальные пакеты\footnote{\href{https://www.debian.org/doc/debian-policy/ch-binary.html\#s-virtual-pkg}{Виртуальные пакеты в руководстве Debian. Дата посещения \DTMdate{2023-12-14}}}, которые предоставляются пакетом \textbf{A}.
	\item \textit{Replaces}.\newline
	      Если пакет \textbf{B} указан в этом поле для пакета \textbf{A}, то \textbf{B} должен быть удалён, прежде чем будет установлен \textbf{A}.
\end{enumerate}

Понимание того, какие взаимоотноешния встречаются между пакетами, позволило однозначно истолковывать те ситуации, когда поля метаданных в различных дистрибутивах не совпадают.
Так, для пакетов формата \texttt{rpm} поле \textit{Depends} встречается в виде \textit{Requires}, но по своей сути означает одно и то же.

\subsection{Выбор дистрибутивов}
Выбраны две операционые системы, чьи особенности на первичном этапе учтены при разработке.
Обе ОС поддерживают \textsc{RISC-V} и входят в десятку самых популярных, согласно данным портала \textsc{DistroWatch}\footnote{\href{https://distrowatch.com/}{Главная страница портала DistroWatch с рейтингами дистрибутивов. Дата посещения \DTMdate{2023-12-14}}}.
Релизы (англ. --- \textbf{Release}) дистрибутивов и так называемые \textbf{upstream} репозитории, содержащие пакеты и их метаданные, кроме \textsc{RISC-V}, выбраны произвольно, так как их выбор не влияет на функциональность разработанной утилиты.
Учтена возможность расширить приложение добавлением других релизов, репозиториев и дистрибутивов.

\paragraph{\ubuntu}
Так как разработка проекта велась на системе, на которой установлен {\ubuntu}, синтаксис метаданных этого дистрибутива представляет собой упрощенный вариант обобщенного синтаксиса {\debian}, а сами метаданные в \texttt{upstream} репозитории лежат в текстовом виде, то он был выбран в качестве первого.
Внедрение {\ubuntu} помогло создать прототип требуемого приложения и консольный интерфейс для конечного пользователя.

\paragraph{\fedora}
Вторым дистрибутивом выбрана {\fedora}, потому что метаданные для этой системы в \texttt{upstream} репозитории предоставлены в виде базы данных \textsc{SQLite}, что существенно упростило процесс внедрения.
Кроме того {\fedora} использует пакеты \texttt{rpm}, в то время как {\ubuntu} --- \texttt{deb}. Таким образом, выбор данного дистрибутива позволил создать приложение, способное обращаться с несовместимыми форматами метаданных различных дистрибутивов.

Проведенный обзор инструментов, формата метаданных и выбор дистрибутивов помогли спроектировать требуемый инструмент.

\subsection{Использованные технологии}
Написание утилиты на языке {\python} входило в производственное задание.
Для поддержания единого стиля кода использован форматтер \textbf{black}\footnote{\href{https://black.readthedocs.io/en/stable/}{Главная страница документации black}} и линтер \textbf{ruff}\footnote{\href{https://docs.astral.sh/ruff/}{Главная страница документации линтера ruff}}, выбранный как альтернатива \textbf{flake8}\footnote{\href{https://flake8.pycqa.org/en/latest/}{Главная страница документации линтера flake8}}.
\texttt{Ruff} применяет тот же набор правил, что и \texttt{flake8}, и по сути ничем не отличается, но не требует дополнительной настройки для совместимости с \texttt{black}.
Кроме того, было принято решение использовать статическую типизацию функций, которая соответствует \enquote{best practices}, для чего подключен \textbf{mypy} с аргументом \textbf{strict}.

Для управления сборкой и зависимостями проекта выбран \texttt{Poetry}\footnote{\href{https://python-poetry.org/}{Главная страница Poetry}}. Этот выбор обладает рядом следующих преимуществ.
\begin{itemize}
	\item \textbf{Декларативное управление зависимостями}. Зависимости и их версии описываются в файле \textbf{pyproject.toml}. \texttt{Poetry} самостоятельно следит, чтобы указанные библиотеки были совместимы друг с другом.
	\item \textbf{Виртуальное окружение}. Автоматически создается виртуальное окружение для проекта, изолируются его зависимости от системных.
	\item \textbf{Интеграция с инструментами сборки и публикации}. \texttt{Poetry} поддерживает не только управление зависимостями и сборку проекта, но также включает в себя функциональность для создания, установки и публикации пакетов.
\end{itemize}

Далее рассмотрены детали реализации
