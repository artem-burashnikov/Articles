% !TeX spellcheck = ru_RU
% !TEX root = burashnikov_depinspect.tex

\section{Детали реализация}
В данном разделе приведены сведения об использованных технологиях и основных библиотеках.
Затроные вопросы, касаемые архитектуры приложения, а также отмечены нетривиальные моменты и решения.

\subsection{Использованные технологии}
Написание утилиты на языке {\python} входило в производственное задание.
Для поддержания единого стиля кода использован форматтер \textbf{black}\footnote{\href{https://black.readthedocs.io/en/stable/}{Главная страница документации black}} и линтер \textbf{ruff}\footnote{\href{https://docs.astral.sh/ruff/}{Главная страница документации линтера ruff}}, выбранный как альтернатива \textbf{flake8}\footnote{\href{https://flake8.pycqa.org/en/latest/}{Главная страница документации линтера flake8}}.
\texttt{Ruff} применяет тот же набор правил, что и \texttt{flake8}, и по сути ничем не отличается, но не требует дополнительной настройки для совместимости с \texttt{black}.
Кроме того, было принято решение использовать статическую типизацию функций, которая соответствует \enquote{best practices}, для чего подключен \textbf{mypy} с аргументом \textbf{strict}.

Для управления сборкой и зависимостями проекта выбран \texttt{Poetry}\footnote{\href{https://python-poetry.org/}{Главная страница Poetry}}. Этот выбор обладает рядом преимуществ:
\begin{itemize}
	\item \textbf{Декларативное управление зависимостями}. Зависимости и их версии описываются в файле \textbf{pyproject.toml}. \texttt{Poetry} самостоятельно следит, чтобы указанные библиотеки были совместимы друг с другом.
	\item \textbf{Виртуальное окружение}. Автоматически создается виртуальное окружение для проекта, изолируются его зависимости от системных.
	\item \textbf{Интеграция с инструментами сборки и публикации}. \texttt{Poetry} поддерживает не только управление зависимостями и сборку проекта, но также включает в себя функциональность для создания, установки и публикации пакетов.
\end{itemize}

\subsection{Базы данных}
Рассматривался вариант создать приложение, которое ищет информацию о пакете на соответствующем ресурсе в интернете, но такое решение оказалось нецелесообразным ввиду того, что количество запросов заранее не ясно --- внешние сервисы, предоставляющие \texttt{API}, почти всегда имеют ограничение на количество обращений к ним. Кроме этого, работа приложения зависела бы от работы стороннего веб-сервиса, вдобавок --- не все из них внешний \texttt{API} имеют. Так, потребовалось реализовать возможность сохранения данных на диске для последующего переиспользования.
Помимо этого, хотелось уметь менять стилистику запросов к метаданным, добавляя какие бы то ни было дополнительные условия или ограничения. Использование баз данных удовлетворяет всем требованиям. Более того, из обзора сделан вывод, что метаданные {\fedora} лежат в \texttt{upstream} в формате \texttt{.sqlite}, а в {\python} имеется встроенная библиотека для импорта с одноименным названием.

Применение \texttt{SQLite} позволило упростить процесс внедрения метаданных {\fedora}, добавило возможность использовать метаданные, хранящиеся на диске, а не в памяти компьютера или на веб-ресурсе, и позволило реальзовать создание нетривиальных запросы.
Для каждого дистрибутива и релиза используются свои файлы баз данных, причем для {\ubuntu} доступные в рамках релиза архитектуры находятся в одной базе, которая создаётся и заполняется самим приложением, а для {\fedora} для каждой комбинации \enquote{релиз-архитектура} загружается отдельная база и никак не меняется.

Схема таблиц для обоих дистрибутивов едина и представлена на \ref{}.

\subsection{API}
Реализованная утилита предоставляет конечному пользователю консольное приложение.
Написание \texttt{CLI} требует значительно меньше времени, чем написание графического интерфейса, позволяет относительно просто расширять функциональность через добавление новых команд, и соответствут тому интерфейсу, который предоставляют рассмотренные ранее схожие инструменты.
Для внедрения консольного приложения использована библиотека \texttt{Click}\footnote{\href{https://click.palletsprojects.com/en/8.1.x/}{Главная страница библиотеки Click}}, распространяемая под лицензией \texttt{BSD-3-Clause}.
Из преимуществ библиотеки можно отметить исчерпывающую документацию с примерами использования и обширную опциональность, что способствовало реализации простого и понятного интерфейса.

\subsection{Архитектура}
Приложение можно разделить на отдельные, независимые друг от друга компоненты, последовательная работа которых приводит к исполнению введенной пользователем команды, что напоминает архитектурный стиль \enquote{Каналы и фильтры}.
Команда от пользователя обрабатывается компонентой \textbf{CLI}, которая может вызывать все остальные.
Кроме \texttt{CLI} Созданы следующие условные компоненты, объединящие, может быть, несколько модулей и функций:
\begin{itemize}
	\item \textbf{Fetcher}, отвечающая за загрузку метаданных, указанных в конфигурационном файле проекта \texttt{pyproject.toml}.
	\item \textbf{Extractor}, отвечающая за обработку загруженных разного рода архивов, содержащих метаданные.
	\item \textbf{Database}, реализующая абстракцию над запросами в базу данных с помощью языка запросов \texttt{SQL}.
	\item \textbf{Printer}, собравшая функции, реализующие вывод результата в консоль.
\end{itemize}

Созданный \enquote{toolchain} позволяет независимо изменять каждую из компонент, минимально затрагивая остальные.
На рисунке \ref{} изображена диаграмма компонент.

\subsection{Использование классов}
Во время поступления команды от пользователя заранее не известно, к какой базе данных нужно совершить обращение. Далее, для инициализации метаданных {\ubuntu} нужно провести целый набор действий (загрузить архивы, произвести парсинг текста, создать и наполнить базу данных), в то время как для метаданных {\fedora} --- загрузить и распаковать файлы \texttt{.sqlite}.
Таким образом, создан метакласс \textbf{Package}, имеющий набор полей, соответствующих полям метаданных, и декларирующий абстрактные методы, сопоставляемые консольным командам, которые реализованы в наследующихся от \textbf{Package} дочерних классах \textbf{Ubuntu} и \textbf{Fedora}.
Так, метод \texttt{init} у \textbf{Ubuntu} запускает требуемую последовательность из \texttt{fetch}, \texttt{extract}, \texttt{parse}, \texttt{load}, а для \textbf{Fedora} --- те же fetch и extract, но с другими параметрами.
Даная идея применяется для всех созданных пользовательских команд и соответствующих им статических методов.

На рисунке \ref{} изображена диаграмма классов.
