% !TeX spellcheck = ru_RU
% !TEX root = burashnikov_bfs.tex

% Опциональные добавления используемых пакетов. Вполне может быть, что они вам не понадобятся, но в шаблоне приведены примеры их использования.
\usepackage{tikz} % Мощный пакет для создание рисунков, однако может очень сильно замедлять компиляцию
\usetikzlibrary{decorations.pathreplacing,calc,shapes,positioning,tikzmark}

% Библиотека для TikZ, которая генерирует отдельные файлы для каждого рисунка
% Позволяет ускорить компиляцию, однако имеет свои ограничения
% Например, ломает пример выделения кода в листинге из шаблона
% \usetikzlibrary{external}
% \tikzexternalize[prefix=figures/]

\newcounter{tmkcount}

\tikzset{
  use tikzmark/.style={
    remember picture,
    overlay,
    execute at end picture={
      \stepcounter{tmkcount}
    },
  },
  tikzmark suffix={-\thetmkcount}
}

\usepackage{booktabs} % Пакет для верстки "более книжных" таблиц, вполне годится для оформления результатов
% В шаблоне есть команда \multirowcell, которой нужен этот пакет.
\usepackage{multirow}
\usepackage{siunitx} % для таблиц с единицами измерений
\usepackage{algorithm}
\usepackage{amsmath}
\usepackage{url}

\newcommand{\cd}[1]{\texttt{#1}}
\newcommand{\inbr}[1]{\left<#1\right>}

% Для названий стоит использовать \textsc{}
\newcommand{\OCaml}{\textsc{OCaml}}
\newcommand{\miniKanren}{\textsc{miniKanren}}
\newcommand{\BibTeX}{\textsc{BibTeX}}
\newcommand{\vsharp}{\textsc{V$\sharp$}}
\newcommand{\fsharp}{\texttt{F\#}}
\newcommand{\csharp}{\textsc{C$\sharp$}}
\newcommand{\GitHub}{\textsc{GitHub}}
\newcommand{\SMT}{\textsc{SMT}}

\newcolumntype{L}[1]{>{\raggedright\let\newline\\\arraybackslash\hspace{0pt}}m{#1}}
%\newcolumntype{C}[1]{>{\centering\let\newline\\\arraybackslash\hspace{0pt}}m{#1}}
\newcolumntype{R}[1]{>{\raggedleft\let\newline\\\arraybackslash\hspace{0pt}}m{#1}}

%  Команды и пакеты, не используемые в шаблоне, которые тем не менее могут быть полезными.

% \newcolumntype{Y}{>{\centering\arraybackslash}X}

% \usepackage{mathrsfs}

%\lstdefinelanguage{ocaml}{
%    keywords={@type, function, fun, let, in, match, with, when, class, type,
%    nonrec, object, method, of, rec, repeat, until, while, not, do, done, as, val, inherit, and,
%    new, module, sig, deriving, datatype, struct, if, then, else, open, private, virtual, include, success, failure,
%    lazy, assert, true, false, end},
%    sensitive=true,
%    commentstyle=\small\itshape\ttfamily,
%    keywordstyle=\ttfamily\bfseries, %\underbar,
%    identifierstyle=\ttfamily,
%    basewidth={0.5em,0.5em},
%    columns=fixed,
%    fontadjust=true,
%    literate={->}{{$\to$}}3 {===}{{$\equiv$}}1 {=/=}{{$\not\equiv$}}1 
%    {|>}{{$\triangleright$}}3 {\\/}{{$\vee$}}2 
%    {/\\}{{$\wedge$}}2 {>=}{{$\ge$}}1 {<=}{{$\le$}} 1,
%    morecomment=[s]{(*}{*)}
%}

\usepackage{upquote}

\usepackage{color}
\definecolor{bluekeywords}{rgb}{0.13,0.13,1}
\definecolor{greencomments}{rgb}{0,0.5,0}
\definecolor{redstrings}{rgb}{0.9,0,0}

\lstdefinelanguage{pseudoFSharp}%
    {morekeywords={let, new, match, with,, open, module, namespace, type, of, member, % 
    and, for, while, true, false, in, do, begin, end, fun, function, return, yield, try, %
    mutable, if, then, else, cloud, async, static, use, abstract, interface, inherit, finally },
    otherkeywords={ let!, return!, do!, yield!, use!, var, from, select, where, call, by },
    keywordstyle=\color{redstrings},
    sensitive=true,
    tabsize=2,
    comment=[l][\color{greencomments}]{//}
}

\definecolor{codegreen}{rgb}{0,0.6,0}
\definecolor{codegray}{rgb}{0.5,0.5,0.5}
\definecolor{codepurple}{rgb}{0.58,0,0.82}
\definecolor{backcolour}{rgb}{1.0,1.0,1.0}

\lstdefinestyle{codelistingstyle}{
    backgroundcolor=\color{backcolour},   
    commentstyle=\color{codegreen},
    keywordstyle=\color{magenta},
    numberstyle=\small\color{codegray},
    stringstyle=\color{codepurple},
    basicstyle=\ttfamily\small,
    breakatwhitespace=false,         
    breaklines=true,                 
    captionpos=b,                    
    keepspaces=true,                 
    numbers=left,                    
    numbersep=5pt,                  
    showspaces=false,                
    showstringspaces=false,
    showtabs=false,                  
    tabsize=2,
    commentstyle=\color{codegreen}
}

\let\sq=\textquotesingle

\usepackage{hyperref}
\usepackage{qtree}
\usepackage{svg}
\usepackage{booktabs}

\captionsetup[figure]{font=small, labelfont=small}
\captionsetup[table]{font=small, labelfont=small}
