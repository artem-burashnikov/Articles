% !TeX spellcheck = ru_RU
% !TEX root = burashnikov_depinspect.tex

\section{Обзор предметной области}
\label{sec:relatedworks}
Для проектирования требуемой системы с целью выявления желаемой функциональности необходимо ознакомиться с инструментами, предоставляющими анализ пакетов.
Кроме того, требуется сделать выбор первоначальных дистрибутивов и проанализировать форматы, в которых представлены метаданные.
В следующих разделах освещены эти вопросы.

\subsection{Существующие аналоги}
В конкретном дистрибутиве используется своё программное обеспечение, позволяющее провести искомый анализ.
Далее представлены некоторые из них.

\subsubsection{\texttt{Apt}}
\texttt{Apt}\footnote{\href{https://manpages.debian.org/stretch/apt/apt.8.en.html\#SCRIPT_USAGE_AND_DIFFERENCES_FROM_OTHER_APT_TOOLS/}{Документация apt. Дата обращения: \DTMdate{2023-12-13}}} --- это пакетный менеджер, по умолчанию поставляемый вместе с дистрибутивом {\debian} и производными системами, основанными на {\debian}, но разрабатываемыми независимо и имеющими собственную целевую аудиторию и цели\footnote{\href{https://www.debian.org/derivatives/}{Понятие прозводной системы на сайте {\debian}. Дата обращения: \DTMdate{2023-12-13}}}.
Согласно документации, \texttt{apt} является высокоуровневой оболочкой для других инструментов, поэтому для просмотра предоставляемой функциональности нужно обратиться к \texttt{apt-cache}\footnote{\href{https://manpages.debian.org/stretch/apt/apt-cache.8.en.html}{Документация \texttt{apt-cache}. Дата обращения: \DTMdate{2023-12-13}}}, команды которого доступны и для \texttt{apt}.
Следующие возможности утилиты согласуются с целью работы:
\begin{enumerate}
	\item \texttt{depends} отображает список каждой зависимости, которую имеет пакет, а также все возможные другие пакеты, которые могут удовлетворить эту зависимость;
	\item \texttt{rdepends} отображает список пакетов, для которых данный является зависимостью;
	\item \texttt{pkgnames} выводит название каждого пакета, о котором знает \texttt{apt}.
\end{enumerate}

Одним из существенных минусов \texttt{apt} является то, что программу не имеет смысла использовать на других дистрибутивах, потому что менеджер плотно интегрирован с системами, под которые создавался.
Кроме того, по умолчанию нет возможности сравнить пакеты в рамках даже одного дистрибутива.

\subsubsection{\texttt{Pactree}}
\texttt{Pactree}\footnote{\href{https://man.archlinux.org/man/extra/pacman-contrib/pactree.8.en}{Документация pactree. Дата обращения: \DTMdate{2023-12-13}}} --- утилита для дистрибутива \textsc{Arch Linux}, позволяющая выводить на экран деревовидную структуру зависимостей для пакетов.
Из функциональности отмечаются те же пункты, что и для \texttt{apt}. Дополнительно имеется возможность вывода результата запроса в формате \texttt{Graphviz}\footnote{\href{https://graphviz.org/}{Приложение Graphviz для визуализации графов. Дата обращения: \DTMdate{2023-12-13}}} для последующей визуализации полученного графа.

Минус \texttt{pactree} заключается в том, что \textsc{Arch~Linux} не поддерживает архитектуру \textsc{RISC-V}, которая, соглас\-но Бенжамину~В.~Мезгеру (Benjamin~W.~Mezger) и др., набирает популярность и активно развива\-ется~\cite{RISCVSurvey}.

\subsubsection{\texttt{Debtree}}
\texttt{Debtree}\footnote{\href{https://manpages.ubuntu.com/manpages/xenial/man1/debtree.1.html}{Документация debtree на ресурсе {\ubuntu}. Дата обращения: \DTMdate{2023-12-13}}} --- инструмент для {\debian} и производных систем, позиционирующий себя как ``граф зависимостей на стероидах''. Функциональность сосредоточена на выводе графа в той или иной форме.
Выводит результет на языке \texttt{dot}, читаемом утилитой \texttt{Graphviz}. Возможности аналогичны уже обозначенным приложениям, однако обладает дополнительными опциями для построения графов (например, цвет, глубина и т.д.).
Отдельно стоит отметить возможность при вызове команды добавить опциональый аргумент \textbf{arch}, указав архитектуру, зависимость пакета на которой требуется показать.

К сожалению, \texttt{Debtree} можно использовать только на производных {\debian}. Инструмент не позволяет сравнивать в каком бы то виде метаданные пакетов между собой.
Кроме того, несмотря на то, что можно указать архитектуру, без дополнительных системных настроек будет использована архитектура системы, на которой запущено приложение, вне зависимости от того, какая архитектура указана аргументом.

\subsubsection{\texttt{Repology}}
\texttt{Repology}\footnote{\href{https://repology.org/docs/about}{Информация о ресурсе Repology. Дата обращения: \DTMdate{2023-12-13}}} в отличие от уже рассмотренных приложений напрямую не позиционируется как система анализа пакетов.
Проект является аггрегатором репозиториев и смежных ресурсов, предоставляющих установочные архивы. \texttt{Repology} имеет обширный список возможностей для анализа версий библиотек, наличия или отсутствия их в репозиториях.
Согласно описанию, сервис отслеживает и позволяет сравнивать версии пакетов между более чем 120 репозиториями.
Хотя \texttt{Repology} и не предоставляет никакой логики для сравннения зависимостей, особое внимание уделено некоторым аспектам технической реализации проекта в целом, потому что они интересны для проектируемой в рамках семестровой практики системы.

Во-первых, серис представляет собой веб-приложение, разделенное на несколько независимых компонент: \textbf{webapp}\footnote{\href{https://github.com/repology/repology-webapp}{Репозиторий компоненты webapp сервиса Repology. Дата обращения: \DTMdate{2023-12-13}}}, \textbf{updater}\footnote{\href{https://github.com/repology/repology-webapp}{Репозиторий компоненты updater сервиса Repology. Дата обращения: \DTMdate{2023-12-13}}} и \textbf{ruleset}\footnote{\href{https://github.com/repology/repology-webapp}{Репозиторий компоненты ruleset сервиса Repology. Дата обращения: \DTMdate{2023-12-13}}}.
В-вторых, сами метаданные скачиваются из архивов соответствующих дистрибутивам репозиториев (те же репозитории использует, например \texttt{apt}), затем сервис производит парсинг загруженных метаданных и сохранение их в базах данных \textsc{SQLite}.
Указанные особенности реализации оказались интересны для проектируемой системы, так как позволили внедрить модульный дизайн и обеспечить переиспользование полученной из репозиториев информации об архивах.

\subsection{Результаты анализа аналогов}
В таблице~\ref{tbl:tool-comparison} представлены рассмотренные инструменты и основная функциональность каждого из них.

\begin{table}[ht]
	\centering
	\begin{tabular}{|c|c|c|c|c|}
		\hline
		& Apt & Pactree & Debtree & Repology \\
		\hline
		Прямые зависимости & \checkmark & \checkmark & \checkmark & \checkmark \\
		\hline
		Обратные зависимости & \checkmark & \checkmark & \checkmark & \checkmark \\
		\hline
		Сравнение метаданных & \ding{55} & \ding{55} & \ding{55} & \ding{55} \\
		\hline
		Построение графа & \ding{55} & \checkmark & \checkmark & \ding{55} \\
		\hline
		Кросс-платформенность & \ding{55} & \ding{55} & \ding{55} & \checkmark \\
		\hline
	\end{tabular}
	\caption{Сравнение инструментов}
	\label{tbl:tool-comparison}
\end{table}

Таким образом, вывод для пакета его прямых и обратных метаданных --- основная функциональность, присущая всем утилитам такого рода.
Далее, не все рассмотренные приложения могут рисовать полный граф зависимостей и никакая из утилит не может напрямую проводить сравнение метаданных пакетов между различными архитектурами ни в рамках одного дистрибутива, ни между пакетами в рамках различных дистрибутивов.
Так, в основу созданного инструмента заложены следующие особенности:
\begin{enumerate}
	\item сравнение по прямым зависимостям пакетов между различными архитектурами в рамках одного дистрибутива;
	\item вывод имен пакетов и архитектур, для которых можно произвести сравнение;
	\item архитектура приложения должна быть гибкой и позволять расширять фукциональность, используя не только зависимости, но и другие потенциально значимые метаданные;
	\item аггрегация пакетов из различных дистрибутивов;
	\item кросс-платформенность инструмента.
\end{enumerate}

\subsection{Формат метаданных}
Прежде всего необходимо понимать возмножные отношения, используемые между пакетами\footnote{\href{https://www.debian.org/doc/debian-policy/ch-relationships.html}{Руководство Debian. Дата посещения \DTMdate{2023-12-14}}}.
Ниже на примере спецификации формата метаданных в дистрибутиве {\debian} приведен список таких отношений с пояснениями:
\begin{enumerate}
	\item \textit{Depends}.\newline
	      Пакет \textbf{A} не будет настроен, пока все пакеты, перечисленные в этом поле не будут корректно настроены. Пакет может находиться в системе в ``не настроенном'' состоянии.
	\item \textit{Recommends}.\newline
	      Для пакета \textbf{A} в этом поле перечислены пакеты, которые обычно устанавливаются вместе c \textbf{A}.
	\item \textit{Suggests}.\newline
	      Перечисленные в этом поле пакеты связаны с пакетом \textbf{A} и, возможно, могут улучшить его полезность, но установка только \textbf{A} без них разумна и ничем не ограничена.
	\item \textit{Enhances}.\newline
	      Работает как \textit{Suggests}, но в обратную сторону: в этом поле указаны пакеты, для которых пакет \textbf{A} попадает в \textit{Suggests}, потенциально улучшая их полезные свойства.
	\item \textit{Pre---Depends}.\newline
	      Означает почти то же, что и \textit{Depends}, но дополнительно требует завершения установки указанных в поле пакетов до установки пакета \textbf{A}.
	\item \textit{Conflicts}.\newline
	      Пакеты, указанные в этом поле, не могут существовать в системе одновременно с пакетом \textbf{A}.
	\item \textit{Provides}.\newline
	      В этом поле указаны виртуальные пакеты\footnote{\href{https://www.debian.org/doc/debian-policy/ch-binary.html\#s-virtual-pkg}{Виртуальные пакеты в руководстве Debian. Дата посещения \DTMdate{2023-12-14}}}, которые предоставляются пакетом \textbf{A}.
	\item \textit{Replaces}.\newline
	      Если пакет \textbf{B} указан в этом поле для пакета \textbf{A}, то \textbf{B} должен быть удалён, прежде чем будет установлен \textbf{A}.
\end{enumerate}

Понимание того, какие взаимоотноешния встречаются между пакетами, позволяет однозначно истолковывать те ситуации, когда названия полей в других дистрибутивах отличаются.
Так, для пакетов формата \texttt{rpm} поле \textit{Depends} встречается в виде \textit{Requires}, но по своей сути означает одно и то же.

Помимо спецификации отношений важным оказался формат файлов метаданных, выкладываемых в репозиториях дистрибутивов.
Далее приведене некоторые из них:
\begin{itemize}
	\item текстовые файлы, в которых каждая строчка является некоторым полем (встречаются, например, в {\debian} и {\ubuntu});
	\item файлы в формате \texttt{XML}, где метаданные указаны с помощью языка разметки;
	\item файл базы данных \textsc{SQLite}, в которой присутствуют таблицы с обозначенными отношениями.
\end{itemize}

Для разработанного приложения важным оказался именно формат \textsc{SQLite}, так как внедрение баз данных помогло решить проблему переиспользования полученных метаданных и существенно увеличило гибкость утилиты, позволив извлекать информацию с разной степенью сложности с помощью реализации абстракции над языком запросов \textsc{SQL}.

\subsection{Выбор дистрибутивов}
Выбраны две операционые системы, чьи особенности на первичном этапе учтены при разработке.
Обе ОС входят в десятку самых популярных, согласно данным портала \textsc{DistroWatch}\footnote{\href{https://distrowatch.com/}{Главная страница портала DistroWatch с рейтингами дистрибутивов. Дата посещения \DTMdate{2023-12-14}}}.
Релизы (англ. --- \textbf{Release}) дистрибутивов и так называемые \textbf{upstream} репозитории, содержащие пакеты и их метаданные, выбраны произвольно, так как их выбор не влияет на функциональность разработанной утилиты.
Учтена возможность расширить приложение добавлением других релизов, репозиториев и дистрибутивов.

\paragraph{\ubuntu}
Так как разработка проекта велась на системе, на которой установлен {\ubuntu}, а синтаксис метаданных этого дистрибутива представляет собой упрощенный вариант обобщенного синтаксиса {\debian}, то этот дистрибутив был выбран в качестве первого.
Внедрение {\ubuntu} помогло создать прототип требуемого приложения и консольный интерфейс для конечного пользователя.

\paragraph{\fedora}
Вторым дистрибутивом выбрана {\fedora}, потому что метаданные для этой системы предоставляются в виде базы данных \textsc{SQLite}, что существенно упростило процесс внедрения.
Кроме того, синтаксис для пакетов и их метаданных, используемый в {\fedora}, по историческим причинам кардинально отличается от {\debian}. Таким образом, выбор данного дистрибутива позволил создать приложение, способное обращаться с несовместимыми форматами метаданных различных дистрибутивов.

Проведенный обзор инструментов, формата метаданных и выбор дистрибутивов помогли спроектировать требуемый инструмент.
В следующих разделах продемонстрирована используемая архитектура и затронуты детали реализации.
