% !TeX spellcheck = ru_RU
% !TEX root = vkr.tex

\section{Эксперимент}

\subsection{Характеристики оборудования}
Оборудование, на котором были поставлены описанные далее эксперименты, обладает следующими характеристиками:
\subsubsection*{OS and Kernel}
\begin{verbatim}
Operating System: Ubuntu 22.04.2 LTS
          Kernel: Linux 5.19.0-41-generic
\end{verbatim}

\subsubsection*{CPU}
\begin{verbatim}
Architecture:       x86_64
Model name:         AMD Ryzen 5 4500U with Radeon Graphics
Thread(s) per core: 1
Core(s) per socket: 6
Socket(s):          1
CPU max MHz:        2375,0000
CPU min MHz:        1400,0000
L1d cache:          192 KiB (6 instances)
L1i cache:          192 KiB (6 instances)
L2 cache:           3 MiB (6 instances)
L3 cache:           8 MiB (2 instances)
\end{verbatim}

\subsubsection*{GPU}
\begin{verbatim}
Device: [AMD/ATI] Renoir (rev c3)
Memory: 256M
\end{verbatim}

\subsubsection*{RAM}
\begin{verbatim}
Memory:      7304
Swap memory: 2047
Total:       9351 
\end{verbatim}

\subsection{Исследовательские вопросы}
Анализ поставленных задач позволил выдвинуть следующие гипотезы:
{\parindent0pt
    
    \subsubsection*{RQ1}
    \label{RQ1}
    Ожидается, что в параллельной версии алгоритма обхода в ширину производительность будет значительно превышать последовательную версию на сильно разреженных неориентированных графах, потому что в таких графах большинство вершин имеют небольшую степень, что позволит эффективно распределить работу между потоками и уменьшить накладные расходы на синхронизацию. Таким образом, параллельная версия должна продемонстрировать ощутимое ускорение.
}
{\parindent0pt
    \subsubsection*{RQ2}
    \label{RQ2}
    Предполагается существование оптимального количества потоков в параллельной версии алгоритма, которое приведет к наибольшему выигрышу в производительности за счет эффективного использования доступных ресурсов вычислительной системы.
}

\subsection{Использованные метрики}

Для исследования~\hyperref[RQ1]{RQ1} решено замерять ускорение (Speedup) параллельной версии алгоритма относительно последовательной со следующим набором контролируемых параметров:
\begin{itemize}
    \item количество вершин в графе;
    \item плотность графа;
    \item количество используемых потоков.
\end{itemize}

Для поиска оптимального значения, обозначенного в~\hyperref[RQ2]{RQ2}, будет проанализировано среднее время работы параллельной версии алгоритма на сильно разреженных графах с использованием разного количества потоков: 1, 2, 4, 8, 16. Кроме того, зафиксируем распределение памяти и нагрузку на потоки во время исполнения.

Все замеры будут выполнены с использованием библиотеки для измерения производительности \texttt{BenchmarkDotNet}~\cite{benchTool}, разрабатываемой и поддерживаемой для платформы \texttt{.NET}.

\subsection{Набор данных}
Для фиксации исследуемых величин были выбраны TODO различных разреженных квадратных матриц из коллекции университета Флориды~\cite{matrixData}.
Плотные матрицы было решено генерировать. Необходимо учесть, чтобы матрица смежности графа заполнялась значениями меток равномерно по всей площади двумерной сетки, потому как группировка ребёр вокруг определенного квадранта матрицы может привести к нежелательным последствиям из-за особенности внутреннего представления матриц в виде деревьев. Для генерации матрицы смежности создавалась двумерная таблица, необходимое количество случайно выбранных ячеек которой заполнялось различными значениями. Функция выбирала ячейки с равномерным распределением.

Информация о выбранных данных представлена в таблице~\ref{table:sparse_matrices}. 
Для обозначения числа ненулевых элементов используется аббревиатура \textit{Nnz}. В таблице приведено название матрицы (при наличии --- официальное), количество строк, количество ненулевых элементов, отношение ненулевых элементов к числу всех возможных элементов (разреженность).

\begin{table}[h]
\begin{center}
\caption{Разреженные матричные данные}
\label{table:sparse_matrices}
\rowcolors{2}{black!2}{black!10}
\scalebox{1.0}{
\begin{tabular}{|l|r|r|r|}
\hline
Матрица & Количество cтрок $R$ & Nnz $M$ & Nnz/$R^{2}$ \\
\hline
\hline
%wing             &    62,032      &   243,088    & 3.9 \\
%luxembourg\_osm  &   114,599      &   239,332    & 2.0 \\
%amazon0312       &   400,727      & 3,200,400    & 7.9 \\
%amazon-2008      &   735,323      & 5,158,388    & 7.0 \\
%web-Google       &   916,428      & 5,105,039    & 5.5 \\
%roadNet-PA       & 1,090,920      & 3,083,796    & 2.8 \\
%roadNet-TX       & 1,393,383      & 3,843,320    & 2.7 \\
%belgium\_osm     & 1,441,295      & 3,099,940    & 2.1 \\
%roadNet-CA       & 1,971,281      & 5,533,214    & 2.8 \\
%netherlands\_osm & 2,216,688      & 4,882,476    & 2.2 \\ 
\hline
\end{tabular}
}
\end{center}
\end{table}


\subsection{Постановка эксперимента}
%Результаты понятно что такое. Тут всякие таблицы и графики, как в таблице \ref{time_cmp_obj_func}. Обратите внимание, как цифры выровнены по правому краю, названия по центру, а разделители $\times$ и $\pm$ друг под другом.

%Скорее всего Ваши измерения будут удовлетворять нормальному распределению, в идеале это надо проверять с помощью критерия Кол\-могорова и т.п.
%Если критерий этого не подтверждает, то у Вас что-то сильно не так с измерениями, надо проверять кэши процессора, отключать Интернет во время измерений, подкручивать среду исполне\-ния (англ. runtime), что\-бы сборка мусора не вмешивалась и т.п.
%Если критерий удовлетворён, то необходимо либо указать мат. ожидание и доверительный/предсказы\-вающий интервал, либо написать, что все измерения проводились с погрешностью, например, в 5\%.
%Замечание: если у вас получится улуч\-шение производительности в пределах погреш\-ности, то это обязательно вызовет вопросы.

%В этом разделе надо также коснуться Research Questions.

%\subsubsection{RQ1} Пояснения
%\subsubsection{RQ2} Пояснения

%\begin{table}
%\def\arraystretch{1.1}  % Растяжение строк в таблицах
%\setlength\tabcolsep{0.2em}
%\centering
% \resizebox{\linewidth}{!}{%
%    \caption{Производительность какого-то алгоритма при различных разрешениях картинок  (меньше~--- лучше), в мс.,  CI=0.95. За пример таблички кидаем чепчики в честь Я.~Кириленко}
%    \begin{tabular}[C]{
%    S[table-format=4.4,output-decimal-marker=\times]
%    *4{S
%          [table-figures-uncertainty=2, separate-uncertainty=true, table-align-uncertainty=true,
%          table-figures-integer=3, table-figures-decimal=2, round-precision=2,
%          table-number-alignment=center]
%          }
%    }
%    \toprule
%        \multicolumn{1}{r}{Resolution} & \multicolumn{1}{r}{\textsc{TENG}} & \multicolumn{1}{r}{\textsc{LAPM}} &
%        \multicolumn{1}{r}{\textsc{VOLL4}} \\ \midrule
%        1920.1080 & 406.23 \pm 0.94 & 134.06 \pm 0.35 & 207.45 \pm 0.42  \\ \midrule
%        1024.768  & 145.0 \pm 0.47  & 39.68 \pm 0.1   &  52.79  \pm 0.1 \\ \midrule
%        464.848   & 70.57 \pm 0.2   & 19.86 \pm 0.01 %    & 32.75  \pm 0.04 \\ \midrule
%        640.480   & 51.10 \pm 0.2   & 14.70 \pm 0.1 & 24  \pm 0.04 \\ \midrule
%        160.120   & 2.4 \pm 0.02    & 0.67 \pm 0.01      & 0.92  \pm 0.01 \\
%        \bottomrule
%    \end{tabular}%
%}
%    \label{time_cmp_obj_func}
%\end{table}

%\clearpage
%% !TeX spellcheck = ru_RU
% !TEX root = vkr.tex

\newcolumntype{C}{ >{\centering\arraybackslash} m{4cm} }
\newcommand\myvert[1]{\rotatebox[origin=c]{90}{#1}}
\newcommand\myvertcell[1]{\multirowcell{5}{\myvert{#1}}}
\newcommand\myvertcelll[1]{\multirowcell{4}{\myvert{#1}}}
\newcommand\myvertcellN[2]{\multirowcell{#1}{\myvert{#2}}}


\afterpage{%
    \clearpage% Flush earlier floats (otherwise order might not be correct)
    \thispagestyle{empty}% empty page style (?)
    \begin{landscape}% Landscape page
        \centering % Center table

\begin{tabular}{|c|c|c|c|c|c|c|c|c|c|c|c|c|c|c|c|c|c|}\hline
%& \multicolumn{17}{c|}{} \\ \hline
\multirowcell{2}{Код модуля \\в составе \\ дисциплины,\\практики и т.п. }
  &\myvertcellN{2}{Трудоёмкость\quad}
  & \multicolumn{10}{c|}{\tiny{Контактная работа обучающихся с преподавателем}}
  & \multicolumn{5}{c|}{\tiny{Самостоятельная работа}}
  & \myvertcellN{2}{\tiny Объем активных и интерактивных\quad}
  \\ \cline{3-17}

&& \myvertcellN{2}{лекции\quad}
    &\myvertcellN{2}{семинары\quad}
    &\myvertcellN{2}{консультации\quad}
    &\myvertcellN{2}{\small практические  занятия\quad}
    &\myvertcellN{2}{\small лабораторные работы\quad}
    &\myvertcellN{2}{\small контрольные работы\quad}
    &\myvertcellN{2}{\small коллоквиумы\quad}
    &\myvertcellN{2}{\small текущий контроль\quad}
    &\myvertcellN{2}{\small промежуточная аттестация\quad}
    &\myvertcellN{2}{\small итоговая аттестация\quad}

    &\myvertcellN{2}{\tiny под руководством    преподавателя\quad}
    &\myvertcellN{2}{\tiny в присутствии     преподавателя\quad}
    &\myvertcellN{2}{\tiny с использованием    методических\quad}
    &\myvertcellN{2}{\small текущий контроль\quad}
    &\myvertcellN{2}{\makecell{\small промежуточная \\ аттестация}}
    &     \\
&& &&&&&&&&& &&&&&&\\
&& &&&&&&&&& &&&&&&\\
&& &&&&&&&&& &&&&&&\\
&&&&&&&&&&& &&&&&&\\
&&&&&&&&&&& &&&&&&\\
&&&&&&&&&&& &&&&&&\\ \hline
Семестр 3 & 2 &30  &&&&&&&&2   & &&&18 &&20 &10\\ \hline
          &   &2--42&&&&&&&&2--25& &&&1--1&&1--1&\\ \hline
Итого     & 2 &30  &&&&&&&&2   & &&&18 &&20 &10\\ \hline
\end{tabular}

        \captionof{table}{Если таблица очень большая, то можно её изобразить на отдельной портретной странице. Не забудьте подробное описание, чтобы содержимое таблицы можно было понять не читая весь текст.}
    \end{landscape}
    \clearpage% Flush page
}



\subsection{Анализ результатов}

%Чуть более неформальное обсуждение, то, что сделано. Например, почему метод работает лучше остальных? Или, что делать со случаями, когда метод классифицирует вход некорректно.
