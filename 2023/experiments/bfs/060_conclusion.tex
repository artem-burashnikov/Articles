% !TeX spellcheck = ru_RU
% !TEX root = artem-burashnikov_parallel_bfs_experiments.tex

\section*{Заключение}
В рамках выполнения данной работы проведено экспериментальное исследование, и были получены следующие результаты.
\begin{itemize}
    \item Реализована параллельная и последовательная версии алгоритма обхода в ширину с использованием абстрактных операций линейной алгебры над матрицами и векторами, представленными в памяти как деревья.
    \item Оценено влияние плотности и количества вершин графа на эффективность параллельной версии обхода в ширину. В частности, установлены те параметы, при которых параллельная версия оказывается эффективнее последовательной.
    \item Найдено оптимально значение количества потоков, позволяющее получить максимальное ускорение.
\end{itemize}

Продолжение исследование может включать оценку ускорения параллельной версии обхода в ширину без использования линейной алгебры и представления матрицы как Quadtree.
Также интерес может представлять выявление зависимости связности графа и выбора стартовой вершины (или вершин) на итоговую производительность алгоритма.