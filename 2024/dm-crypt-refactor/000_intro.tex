% !TeX spellcheck = ru_RU
% !TEX root = vkr.tex

\section*{Введение}
% \thispagestyle{withCompileDate}

% Формат из 4х частей рекомендуется в курсе Д.~Кознова~\cite{koznov} по написанию текстов.

% \begin{enumerate}
%     \item Известная информация (background/обзор).
%     \item Неизвестная информация (пробел в знаниях, \enquote{Gap}).
%     \item Гипотезы, вопросы, цели~--- \enquote{что болит}, что будет решать Ваша работа.
%     \item Подход, план решения задачи, предлагаемое решение.
% \end{enumerate}

% Последний абзац должен читаться и быть понятен в отрыве от других трёх.
% Никакие абзацы нумеровать нельзя.

% Части (абзацы) должны занять максимум две страницы, идеально уложиться в одну.

% С.-П. Джонс~\cite{SPJGreatPaper} предлагает несколько другой формат написания введения.
% Вполне возможно, что если Ваша работа про языки программирования, то его формат будет удачнее.

% Введение и заключение читают чаще всего, поэтому они должны быть \enquote{вылизаны} до блеска.

% \blfootnote{
%     Иногда рецензенту полезно знать какого числа компилировался текст, чтобы оценить актуальность версии текста.
%     В этом случае полезно вставлять в текст дату сборки.
%     Для совсем официальных релизов документа это не вполне канон.\\
%     Также здесь имеет смысл указать, если работа сделана на деньги, например, Российского Фонда Фундаментальных Исследований (РФФИ) по гранту номер такой-то, и т.п.}

Ядро Linux можно рассматривать как набор отдельных инфраструктур и подсистем.

Согласно репозиторию разработчиков DMCrypt\footnote{\href{https://gitlab.com/cryptsetup/cryptsetup/-/wikis/DMCrypt}{https://gitlab.com/cryptsetup/cryptsetup/-/wikis/DMCrypt} (дата обращения \DTMdate{2024-09-22})}:
\begin{itemize}
    \item \textbf{\textit{device-mapper}} --- это инфраструктура, которая \enquote{обеспечивает обобщенный способ создания слоёв виртуальных блочных устройств над физическими носителями};
    \item \textbf{device-mapper crypt target} (далее --- \textbf{dm-crypt}) --- это модуль \textbf{crpyt} этой инфраструктуры, предоставляющий \enquote{прозрачное шифрование блочных устройств с использованием kernel crypto API};
\end{itemize}
В настоящий момент \texttt{dm-crypt.c} представляет собой исходный файл, включающий в себя код блочного устройства, код шифрования и код дешифрования.
Улучшение структуры подсистем, составляющих ядро Linux, является актуальной проблемой, поэтому научным руководителем была поставлена задача провести рефакторинг \texttt{dm-crypt.c}, выделив логические компоненты в отдельные модули, обеспечив при этом их взаимодействие.
