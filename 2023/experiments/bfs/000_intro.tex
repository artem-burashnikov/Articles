% !TeX spellcheck = ru_RU
% !TEX root = artem-burashnikov_parallel_bfs_experiments.tex

\section*{Введение}
Использование такой абстракции как \textit{граф} для анализа и изучения различных форм реляционных данных имеет большое значение. Теоретические проблемы, существующие в областях применения, включают в себя определение и выявление значащих объектов, обнаружение аномалий, закономерностей или внезапных изменений, кластеризацию тесно связанных сущностей. Поиск в ширину (анг. --- \textit{Breadth-First Search}, \textit{BFS}) и поиск в глубину (англ. --- \textit{Depth-First Search}, \textit{DFS}) являются двумя основными алгоритмами для cистематического исследование графов. 

Для удовлетворения задач теоретического анализа в современных приложениях важна скорость вычислений. Одним из способов получить ускорение выступают \textit{параллельные вычисления}, дающие выигрыш в производительности на современных многоядерных системах. Благодаря природе своей работы, алгоритм обхода в ширину естественным образом позволяёт внедрять в свою реализацию использование дополнительных потоков и ядер процессора. Таким образом, простота реализации, широкая область применения и возможность использования паралелльных вычислений делают BFS популярным инструментом и объектом исследований.

Нужно отметить, что внедрение дополнительных потоков не всегда может давать существенный прирост производительности, а в каких-то случаях --- значительно замедлить работу алгоритма. Поэтому интерес данной работы представляет параллельная реализации BFS и характеристики графов, для которых она оказывается эффективнее последовательной.

