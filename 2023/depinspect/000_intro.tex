% !TeX spellcheck = ru_RU
% !TEX root = burashnikov_depinspect.tex

\section*{Введение}
% \thispagestyle{withCompileDate}
\label{sec:intro}

Во время работы с операционной системой (сокр. --- ОС) установка, обновление и управление программами являются неотъемлемой частью пользовательского опыта.
В ОС семейства {\linux} эти процедуры происходят с использованием \textbf{пакетов}~\cite{IntroductionToLinux}.
Пакетом называется архив специального формата, который содержит необходимые бинарные и конфигурационные файлы, список действий, выполняемых во время установки, а также любую другую важную информацию, в том числе данные о зависимостях.
При этом в каждый такой архив включен только необходимый минимум компонент, остальные указываются в качестве зависимостей.
Например, в один могут быть вложены бинарные файлы, во второй --- общие библиотеки, в третий --- документация.

{\linux} состоит из ядра и сопутствующего ему программного обеспечения, вместе называемых \textbf{дистрибутивом}.
Несмотря на то, что дистрибутивы формируются примерно из одного и того же набора популярных программ и полезных библиотек, сами пакеты, их формат и зависимости могут различаться.
Например, в {\centos} пакет \textbf{openldap} зависит от \textbf{openssl}, а в {\ubuntu} --- от \textbf{gnutls}.
Кроме того, для одного конкретного дистрибутива набор зависимостей у некоторого пакета может оказаться другим при использовании системы с другой архитектурой процессора, причем особенности могут быть как в компонентах, так и в версиях одной и той же компоненты.

Эти и другие характеристики учитываются при проведении анализа пакетов и их метаданных в {\linux} между разными архитектурами и дистрибутивами.
Некоторыми причинами и целями такого анализа являются следующие пункты:
\begin{enumerate}
	\item \textit{Совместимость и преносимость}. Анализ зависимостей помогает при разработке совместимого и переносимосимого между устройствами программного обеспечения.
	\item \textit{Оптимизация и эффективность}. Оптимизированные версии программ могут быть разработаны для конкретных процессоров. Анализ зависимостей позволяет определить, какие компоненты и где требуют оптимизаций.
	\item \textit{Специализированные решения}. В некоторых случаях различные архитектуры и дистрибутивы могут использоваться для специализированных задач (например, встроенные системы, или высокопроизводительные вычисления). Анализ пакетов способствует адаптации системы к конкретным требованиям и использованию специализированных компонент.
\end{enumerate}

Ввиду отсутствия единого формата архивов и их метаданных, существенных различий между дистрибутивами и используемыми ими пакетными менеджерами проведение глубокого междистрибутивный анализа явяется актуальной проблемой.
В рамках семестровой практики научным руководителем была поставлена задача спроектировать и создать на языке {\python} систему, которая бы помогла её решению.
