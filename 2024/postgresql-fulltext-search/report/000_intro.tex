% !TeX spellcheck = ru_RU
% !TEX root = vkr.tex

\section*{Введение}
% \thispagestyle{withCompileDate}

% Формат из 4х частей рекомендуется в курсе Д.~Кознова~\cite{koznov} по написанию текстов.

% \begin{enumerate}
%     \item Известная информация (background/обзор).
%     \item Неизвестная информация (пробел в знаниях, \enquote{Gap}).
%     \item Гипотезы, вопросы, цели~--- \enquote{что болит}, что будет решать Ваша работа.
%     \item Подход, план решения задачи, предлагаемое решение.
% \end{enumerate}

% Последний абзац должен читаться и быть понятен в отрыве от других трёх.
% Никакие абзацы нумеровать нельзя.

% Части (абзацы) должны занять максимум две страницы, идеально уложиться в одну.

% С.-П. Джонс~\cite{SPJGreatPaper} предлагает несколько другой формат написания введения.
% Вполне возможно, что если Ваша работа про языки программирования, то его формат будет удачнее.

% Введение и заключение читают чаще всего, поэтому они должны быть \enquote{вылизаны} до блеска.

% \blfootnote{
%     Иногда рецензенту полезно знать какого числа компилировался текст, чтобы оценить актуальность версии текста.
%     В этом случае полезно вставлять в текст дату сборки.
%     Для совсем официальных релизов документа это не вполне канон.\\
%     Также здесь имеет смысл указать, если работа сделана на деньги, например, Российского Фонда Фундаментальных Исследований (РФФИ) по гранту номер такой-то, и т.п.}

\textit{Полнотекстовый поиск} (англ. --- \textit{full text search}) --- это метод поиска информации в текстовых данных, позволяющий находить документы, содержащие определенные слова или фразы, с их последующим ранжированием.
В отличие от традиционных методов поиска, таких как поиск по индексам, ключам или применение регулярных выражений, полнотекстовый поиск анализирует документы на более глубоком уровне, используя морфологический анализ, стоп-слова и другие механизмы.

\textbf{PostgreSQL} --- это объектно-реляционная СУБД с открытым исходным кодом.
В ней реализован встроенный механизм, позволяющий эффективно обрабатывать текстовые данные.
Этот механизм включает специализированные типы данных и операторы для быстрого поиска по тексту, при этом поддерживается сортировка результатов и учитываются особенности естественных языков.

Задача разобраться в особенностях механизма полнотекстового поиска в PostgreSQL была поставлена научным руководителем.
Актуальность работы обусловлена растущим объемом текстовой информации в современных цифровых системах, а возможность эффективного поиска в таких данных является важной для множества приложений, требующих использования продвинутых инструментов для работы с текстом.

PostgreSQL широко используется в российских и международных проектах, что делает знакомство с методами полнотекстового поиска, реализованными в ней, важным шагом для разработчиков и администраторов баз данных.
Кроме этого, данная работа может быть использована в образовательных целях, способствуя более глубокому пониманию методов поиска и их применения в реальных проектах.
