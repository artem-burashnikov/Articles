% !TeX spellcheck = ru_RU
% !TEX root = vkr.tex

%\blfootnote{
%	Иногда рецензенту полезно знать какого числа %компилировался текст, чтобы оценить актуальность версии %текста. В этом случае полезно вставлять в текст дату сборки. %Для совсем официальных релизов документа это не вполне канон.\\
%Также здесь имеет смысл указать, если работа сделана на %деньги, например, Российского Фонда Фундаментальных %Исследований (РФФИ) по гранту номер такой-то, и т.п.}

\section*{Введение}
Использование такой абстракции как \textit{граф} для анализа и изучения различных форм реляционных данных имеет большое значение. Теоретические проблемы, существующие в областях применения, включают в себя определение и выявление значащих объектов, обнаружение аномалий, закономерностей или внезапных изменений, кластеризацию тесно связанных сущностей. Решения для этих проблем обычно используют классические алгоритмы, применяемые для графов. Для удовлетворения потребностей теоретического анализа в современных приложениях важно ускорить решение задач, лежащих в основе этих алгоритмов. Одним из способов такого ускорения выступают \textit{параллельные вычисления}, являющееся предпочтительной стратегией, дающей выигрыш в производительности на современных многоядерных системах.

Систематическое исследование всех ребер и вершин графа называется его обходом. Нужно отметить, что размеры графов, возникающих сегодня, массивны, но такие графы одновременно могут быть сильно разреженными, то есть иметь малое число рёбер по сравнению с количеством вершин. Для эффективного хранения таких объектов в памяти компьютера используются различные вспомогательные структуры, позволяющие существенно сэкономить занимаемое место.

Распараллеливание не всегда даёт существенный прирост производительности, а в каких-то случаях может значительно замедлить работу алгоритма. Поэтому интерес исследования представляют такие параметры параллельной реализации BFS и характеристики графов, для которых она оказывается эффективнее последовательной. Также отдельное влияние на производительность оказывают накладные расходы на создание самих параллельных задач, отвечающих за отдельные асинхронные вычисления. Указанное в совокупности определяет, какая версия (последовательная или параллельная) предпочтительнее к использованию при том или ином сценарии.
