% !TeX spellcheck = ru_RU
% !TEX root = artem-burashnikov_parallel_bfs_experiments.tex

\section{Постановка задачи}
\label{sec:task}
\noindent Целью работы является проведение экспериментального исследования производительности обхода в ширину и анализ следующих исследовательских вопросов. 

\begin{itemize}
    \item{\textbf{RQ1}\label{rq1}} \newline
    При каких параметрах графа выгоднее использовать параллельную версию алгоритма, а при каких последовательную?
    \item{\textbf{RQ2}\label{rq2}} \newline
    Использование какого количества потоков даёт наибольший выигрыш в производительности и почему?
\end{itemize}

\noindent Для выполнения поставленной цели были сформулированы перечисленные ниже задачи.
 \begin{itemize}
 	\item Реализовать параллельную и последовательную версии алгоритма обхода в ширину с использованием структур, подходящих для хранения в памяти компьютера разреженных матриц и векторов.
 	\item  Оценить влияние конкретных характеристик графа и количества используемых потоков на итоговую производительность BFS и найти критические величины указанных параметров, при которых она максимальна.
 \end{itemize}

\noindent Выполнение обозначенных задач позволило определить какая версия алгоритма (последовательная или параллельная) предпочтительнее к использованию при том или ином сценарии.
